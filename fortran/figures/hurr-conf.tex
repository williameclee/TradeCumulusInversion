\documentstyle[twocolumn,epsf]{amsconf}
%\input epsf
\setlength{\textwidth}{165mm}
\parindent.5in
\columnsep13mm
\setlength{\textheight}{9.0in}
\setlength{\hoffset}{-.450in}
\pagestyle{empty}
\setlength{\voffset}{-.9in}          
\widowpenalty10000 
\clubpenalty150
\begin{document}
\def \0degC{0$^\circ$C}
\def\jas{{\it J. Atmos. Sci.}, $\;\;$}
\def\mwr{{\it Mon. Wea. Rev.}, $\;\;$}
\def\qjrms{{\it Quart. J. Roy. Meteorol. Soc.}, $\;\;$}
\def\vol #1 {{\bf #1}, $\;\;$}
\baselineskip4.25mm
\title{\begin{flushleft} {\bf 14A.1} \end{flushleft} \vskip -0.20truein 
DYNAMICAL ADJUSTMENT OF THE TRADE WIND INVERSION LAYER}
\author{Wayne H. Schubert\thanks{Corresponding Author address:  Wayne Schubert, 
Department of Atmospheric Sciences, Colorado State University, Fort Collins, CO  80523}, 
Paul E. Ciesielski, and Richard H. Johnson\\
\\
Colorado State University\\
Fort Collins, Colorado}

\maketitle

\thispagestyle{empty}

\section{INTRODUCTION}

     Most of the latent heat release and rising motion of the tropical
atmosphere occurs in the ITCZ and its associated wave disturbances.  The area
involved in this rising motion is a small percentage of the total area of the
tropics.  Most of the tropical/subtropical region is under the influence of
subsidence.  In the subsidence regions stratocumulus and shallow cumulus
convection is capped by the stable trade-wind inversion. As is
often done, we use the term ``inversion" in a generic sense to include
not only layers in which $\partial T/\partial z > 0$, but also layers of
enhanced stability in which $\partial T/\partial z$ is slightly negative.

In schematic north-south cross sections the trade inversion layer is
often depicted as sloping upward as air flows toward the ITCZ.  This
conceptual view is consistent with {\it purely thermodynamic} boundary layer
models, which predict a deeper boundary layer with increasing sea-surface
temperature and decreasing large-scale subsidence.  The slopes implied by such
thermodynamic models and incorporated into schematic diagrams are
approximately
1500 m/1000 km.  In contrast, observational studies of the inversion structure
over the Atlantic and Pacific reveal a less dramatic slope, on the order of
250 m/1000 km.  

To address this inconsistency, we adopt a somewhat different
view of the trade inversion layer.  In particular, rather than regarding it as
a purely thermodynamic structure, we regard it as a dynamical structure.  By
formulating a generalization of the Rossby adjustment problem, we investigate
the dynamical adjustments of a trade-wind inversion layer of variable strength
and depth.  From the solution of the adjustment problem there emerges the
notion that the subtropics control the inversion structure in the tropics,
i.e., that the subtropical inversion height is dynamically extended into the
tropics in such a way that there is little variation in the depth of the
boundary layer.

\section{OBSERVATIONAL STUDIES}

\begin{figure}[t]

\centerline{
\epsfxsize=2.9 in
\epsfysize=2.9 in
\epsffile{fig1.eps}}

\caption{Cumulative frequency (percent) of lapse rates greater than $-4$ K
km$^{-1}$ (solid), $0$ K km$^{-1}$ (dashed) and $4$ K km$^{-1}$
(dotted) at Porto Santo Island (33.1$^\circ$N, 16.3$^\circ$W.)}

\end{figure}

\begin{figure}[t]

\centerline{
\epsfxsize=2.9 in
\epsfysize=2.9 in
\epsffile{fig2.eps}}

\caption{Cumulative frequency (percent) of lapse rates greater than $-4$ K
km$^{-1}$ (solid), $0$ K km$^{-1}$ (dashed) and $4$ K km$^{-1}$
(dotted) for San Nicolas Island ($33.3^\circ$N, $119.6^\circ$W).}

\end{figure}

\begin{figure}[t]

\centerline{
\epsfxsize=2.9 in
\epsfysize=2.9 in
\epsffile{fig3.eps}}

\caption{Cumulative frequency (percent) of lapse rates greater than $-4$ K
km$^{-1}$ (solid), $0$ K km$^{-1}$ (dashed) and $4$ K km$^{-1}$
(dotted) for Kwajelein (8.7N, $167.7^\circ$E).}

\end{figure}

Figure 1 shows the frequency of trade inversions at the subtropical 
station Porto Santo, where the cloudiness is a mixture of stratocumulus and shallow
cumulus. This
figure is based on 203 high vertical resolution (10--15 m) radiosonde
soundings taken between 1 and 28 June 1992.  
These diagrams indicate that the
most probable trade inversion height at Porto Santo is slightly below the 800
mb ($\approx 2000$ m)  level, with a tendency for the stronger inversions to
be somewhat lower and near 850 mb ($\approx 1500$ m).
For comparison the same analysis was performed on 70 high resolution
radiosonde soundings taken between 30 June and 20 July 1987 at 
San Nicolas Island.  The results of
this analysis are shown in Fig.~2.  At San Nicolas the sea surface is colder
and the mean divergence is larger than at Porto Santo, resulting in a
persistent stratocumulus regime rather than the mixed stratocumulus/shallow
cumulus regime observed at Porto Santo.  Note that the inversions at San
Nicolas tend to be lower and stronger than those at Porto Santo.  For example,
at San Nicolas the highest probability of strong inversions
($\partial T/\partial z > 4.0$ K km$^{-1}$) occurs at 915 mb ($\approx 850$ m)
and is 24\%, while at Porto Santo the highest probability of strong inversions
occurs at 860 mb ($\approx 1400$ m) and is 15\%.

     Evidence concerning the large-scale structure of the trade-wind boundary
layer has been slowly accumulating for much of this century.  Some of the
earliest data was acquired from the 1925--27 expedition of the German research
vessel Meteor I.  During this expedition, 217 kite soundings of the Atlantic
trade-wind boundary layer were obtained.  Based on this data, von Ficker
(1936) produced the first map of the height of the trade-wind inversion base
over the Atlantic.  This map (not shown), which can be regarded as typical of the
northern summer, depicts low inversions over the cold ocean currents near
northwest and southwest Africa.  These two regions of low inversions are
similar in many respects to San Nicolas Island.  Also shown in von Ficker's map
is a fairly
flat inversion at approximately 2000 m over much of the equatorial region.
Later Atlantic observations support the general picture given by von Ficker.

     Gutnick (1958) extended the study of the trade-wind inversion into the
Caribbean by analyzing three years of radiosonde soundings for six stations
(approximately 3800 total soundings). His analysis reveals a trade-wind inversion base at
approximately 2200 m over the Caribbean, with a gentle upward slope
($\approx$ 250 m/1000 km) toward the equator.

     A depiction of the trade-wind inversion structure over the tropical
eastern Pacific was provided by Firestone and Albrecht (1986) and Betts and
Albrecht (1987), who analyzed dropwindsonde data from approximately 900
soundings made on flights between Hawaii and the equator during the first
special observing period (15 January to 20 February 1979) of FGGE. 
An important conclusion of this work was that there is only a
slight increase in the height of the trade inversion ($\approx$ 300 m/1000 km)
toward the equator over the eastern Pacific.

     In an extension of the previous two studies, Kloesel and Albrecht (1989) 
analyzed additional dropwindsonde soundings made on flights south of Acapulco
and southwest of the Panama Canal Zone during the second special observing
period (10 May to 8 June 1979) of FGGE.  Based on this extended data set,
Kloesel and Albrecht concluded that the trade inversion height tends to be
uniform over the entire tropical eastern Pacific from 5$^\circ$S to
15$^\circ$N and from 90$^\circ$W to 170$^\circ$E. 

    Preliminary analysis of TOGA/COARE sounding data has shown
that trade-wind inversion heights over the
western Pacific tend to be horizontally uniform and near 800 mb, just as in
the eastern Pacific and in the Atlantic.  For example, Fig.~3 shows
the results of a statistical analysis of 125 soundings made from Kwajelein
at 00 UTC during November 1992--February 1993. 
The lapse rate statistics for Kwajelein are similar to those at Porto Santo
(Fig.~1) with a most probable inversion height of 800 mb.

\section{PREVIOUS THEORETICAL STUDIES}
 
     Two theoretical views of the trade inversion are as follows.  The first
(e.g., Lilly 1968) considers the
trade inversion as a pure thermodynamical phenomenon which can be modeled
using steady state, horizontally homogeneous assumptions. In such
one-dimensional thermodynamic models of stratocumulus and shallow cumulus the
equilibrium trade wind boundary layer depth is determined by a balance between
the radiative and moist convective processes tending to deepen the layer and
the large-scale subsidence tending to shallow the layer. This view leads to
the notion that the boundary layer depth and thermodynamic structure are
controlled by local values of sea surface temperature and large-scale
divergence.  

The second view (e.g., Albrecht 1984) also
involves pure thermodynamical, steady state arguments, but recognizes the
importance of horizontal inhomogeneities.  In this view, horizontal advection
is important and entrainment can occur not only by large-scale subsidence but
also by horizontal flow across a sloping inversion.  This view leads to the
notion that the boundary layer depth and thermodynamic structure are
controlled by a weighted average of the sea surface temperature and
large-scale divergence upstream along the trajectory.  Both of these views 
predict a deepening boundary layer along trajectories moving toward the ITCZ
over increasing sea surface temperature and decreasing large-scale subsidence.
 In possession of concepts like these, it is natural to construct schematic
cross sections which depict the trade inversion
as sloping upward from very near the sea surface in the subtropics to the
midtroposphere near the edge of the ITCZ.  The slopes implied by such
thermodynamic models and incorporated into schematic diagrams are
approximately 1500 m/1000 km, in contrast to observed values of 250 m/1000 km.

\section {DYNAMICAL/THERMODYNAMICAL APPROACH}

     In an effort to understand why the observed trade-wind inversion layer is
so flat compared to the predictions of pure thermodynamic models, we shall
adopt a third view, one that involves combined dynamical/thermodynamical
arguments. Thus, let us consider the following problem.  If thermodynamic
processes tend to deepen the trade-wind boundary layer toward the equator,
will dynamical adjustment processes oppose this effect and restore a nearly
flat trade wind inversion height?  We can think of this as a generalized
version of the classic Rossby adjustment problem.  At the conference 
we will present the results of idealized experiments which illustrate two
concepts---the dynamical extension of subtropical inversions into the tropics
and the dynamical adjustment of sloping trade-wind inversion layers.  

Model results indicate that the trade inversion layer is strongly
coupled horizontally.  Rapid variations in trade inversion height are
dynamically smoothed.  In this way, the trade inversion height is controlled
by horizontally averaged (over a Rossby length or more) values of the
sea-surface temperature, divergence and above-inversion atmospheric structure.

\section{ACKNOWLEDGEMENTS} 

This research has been supported by the Office of Naval Research under  
grant N00014-91-J-1422.

\section{REFERENCES}

\begin{references}

%Albrecht, B. A., 1979:  A model for the thermodynamic structure of the
%trade-wind booundary layer. II: Applications.  {\sl J.\ Atmos.\ Sci.},
%{\bf 36}, 90--98.

%Albrecht, B. A., 1981:  Parameterization of trade-cumulus cloud amount. 
%{\sl J.\ Atmos.\ Sci.}, {\bf 38}, 73--89.

Albrecht, B. A., 1984:  A model study of downstream variations of the
thermodynamic structure of the trade winds.  {\sl Tellus},
{\bf 36A}, 187--202.

%Albrecht, B. A., A. K. Betts, W. H. Schubert and S. K. Cox, 1979:  A model for
%the thermodynamic structure of the trade-wind boundary layer. I: Theoretical
%formulation and sensitivity tests.  {\sl J.\ Atmos.\ Sci.},
%{\bf 36}, 73--89.

%Betts, A. K., 1973: Nonprecipitating cumulus convection and its
%parameterization. {\sl Quart.\ J.\ Roy.\ Meteor.\ Soc.}, {\bf 99}, 178--196.

Betts, A. K., and B. A. Albrecht, 1987:  Conserved variable analysis of the
convective boundary layer thermodynamic structure over the tropical oceans. 
{\sl J.\ Atmos.\ Sci.}, {\bf 44}, 83--99.

%Bretherton, C. S., 1993:  Understanding Albrecht's model of trade cumulus
%cloud fields.  {\sl J.\ Atmos.\ Sci.}, {\bf 50}, 2264--2283.

%Charney, J. G., and M. E. Stern, 1962:  On the stability of internal
%baroclinic jets in a rotating atmosphere.  {\sl J. Atmos.\ Sci.}, {\bf 19},
%159--172.

%Eliassen, A., 1983: The Charney--Stern theorem on barotropic-baroclinic
%instability.  {\sl Pure Appl.\ Geophys.}, {\bf 121}, 563--572.

Ficker, H. von, 1936: Die passatinversion. {\sl Ver\"of.\ des Meteor.\ Inst.\
der Univ.\ Berlin}, {\bf 1}, No.~4.

Firestone, J. K., and B. A. Albrecht, 1986:  The structure of the atmospheric
boundary layer in the central equatorial Pacific during January and February
of FGGE.  {\sl Mon.\ Wea.\ Rev.}, {\bf 114}, 2219--2231.

Gutnick, M., 1958: Climatology of the trade-wind inversion in the Caribbean.
{\sl Bull.\ Amer.\ Meteor.\ Soc.}, {\bf 39}, 410--420.

%Gutzler, D. S., 1993: Uncertainties in climatological tropical humidity %profiles: Some implications for estimating the greenhouse effect. {\sl J.\ %Climate}, {\bf 6}, 978--982.

%Hack, J. J., and W. H. Schubert, 1990:  Some dynamical properties of idealized
%thermally-forced meridional circulations in the tropics.  {\sl Meteorology and
%Atmospheric Physics}, {\bf 44}, 101--117.

%Hack, J. J.,  W. H. Schubert, D. E. Stevens and H.-C. Kuo, 1989:  Response of
%the Hadley circulation to convective forcing in the ITCZ.
%{\sl J.\ Atmos.\ Sci.}, {\bf 46}, 2957--2973.

%Haraguchi, P. Y., 1968: Inversions over the tropical eastern Pacific Ocean.
%{\sl Mon.\ Wea.\ Rev.}, {\bf 96}, 177--185.

%Holland, J. Z., and E. M. Rasmusson, 1973:  Measurements of the atmospheric
%mass, energy, and momentum budgets over a 500-kilometer square of tropical
%ocean.  {\sl Mon.\ Wea.\ Rev.}, {\bf 101}, 44--55.

%Hoskins, B. J., M. E. McIntyre and A. W. Robertson, 1985: On the use and
%significance of isentropic potential vorticity maps.
%{\sl Quart.\ J.\ R.\ Met.\ Soc.}, {\bf 111}, 877--946.

%Johnson, R. H., 1993:  Midlatitude convective systems.  NCAR Colloquium on %Clouds and Climate.

%Johnson, R. H., J. F. Bresch, P. E. Ciesielski and W. A. Gallus, 1993: The
%TOGA/COARE atmospheric sounding array: Its performance and preliminary
%scientific results.  Preprints of the 20$^{\rm th}$ Conference on Hurricanes
%and Tropical Meteorology, San Antonio, Texas, Amer.\ Meteor.\ Soc., 1--4.

Kloesel, K. A., and B. A. Albrecht, 1989:  Low-level inversions over the
tropical Pacific---ther\-mo\-dy\-nam\-ic structure of the boundary layer and
the above-inversion moisture structure. {\sl Mon.\ Wea.\ Rev.}, {\bf 117},
87--101.

Lilly, D. K., 1968:  Models of cloud-topped mixed layers under a strong
inversion. {\sl Quart.\ J.\ Roy.\ Meteor.\ Soc.}, {\bf 94}, 292--309.

%Magnusdottir, G., and W. H. Schubert, 1991:  Semigeostrophic theory on the
%hemisphere. {\sl J. Atmos.\ Sci.}, {\bf 48}, 1449--1456.

%Nitta, T., and S. Esbensen, 1974: Heat and moisture budget analyses using
%BOMEX data.  {\sl Mon.\ Wea.\ Rev.}, {\bf 102}, 17--28.

%Ramage, C. S., S. J. S. Khalsa and B. N. Meisner, 1981:  The central Pacific
%near-equatorial convergence zone. {\sl J.\ Geophys.\ Res.}, {\bf 86},
%6580--6598.

%Randall, D. A., 1980: Entrainment into a stratocumulus layer with distributed
%radiative cooling. {\sl J. Atmos.\ Sci.}, {\bf 37}, 148--159.

%Rogers, R. R., W. L. Ecklund, D. A. Carter, K. S. Gage and S. A. Ethier, 1993:
% Research applications of a boundary-layer wind profiler. {\sl Bull.\ Amer.\ 
%Meteor.\ Soc.}, {\bf 74}, 567--580.

%Sarachik, E. S., 1985: A simple theory for the vertical structure of the
%tropical atmosphere. {\sl PAGEOPH}, {\bf 123}, 261--271.

%Schubert, W. H., P. E. Ciesielski, D. E. Stevens and H.-C. Kuo, 1991:
%Potential vorticity modeling of the ITCZ and the Hadley circulation.
%{\sl J. Atmos.\ Sci.}, {\bf 48}, 1493--1509.

%Schubert, W. H., P. E. Ciesielski, T. B. McKee, J. D. Kleist, S. K. Cox, C. M.
%Johnson-Pasqua, and W. L. Smith Jr., 1987:  Analysis of Boundary Layer
%Sounding Data from the FIRE Marine Stratocumulus Project. Colorado State
%University, Department of Atmospheric Science, Report No. 419, 97 pp.

%Schubert, W. H., S. K. Cox, T. B. McKee, D. A. Randall, P. E. Ciesielski, J.
%D. Kleist, and E. L. Stevens, 1992:  Analysis of Sounding Data from Porto
%Santo Island During ASTEX. Colorado State University, Department of
%Atmospheric Science, Report No. 512, 96 pp.

%Schubert, W. H., J. S. Wakefield, E. J. Steiner and S. K. Cox, 1979a: Marine
%stratocumulus convection. Part I: Governing equations and horizontally
%homogeneous solutions. {\sl J.\ Atmos.\ Sci.}, {\bf 36}, 1286--1307.

%Schubert, W. H., J. S. Wakefield, E. J. Steiner and S. K. Cox, 1979b: Marine
%stratocumulus convection. Part II:  Horizontally inhomogeneous solutions.
%{\sl J.\ Atmos.\ Sci.}, {\bf 36}, 1308--1324.

%Shutts, G. J., 1980:  Angular momentum coordinates and their use in zonal,
%geostrophic motion on a hemisphere. {\sl J. Atmos.\ Sci.}, {\bf 37},
%1126--1132.

%Wakefield, J. S., and W. H. Schubert, 1981:  Mixed-layer model simulation of
%Eastern North Pacific stratocumulus.  {\sl Mon.\ Wea.\ Rev.}, {\bf 109},
%1959--1968.

\end{references}

\end{document}

\section{Dynamical adjustment of the trade inversion layer}

\subsection{Experiment 1: dynamical extension of subtropical inversions into
the tropics}

     Using the results of an idealized adjustment problem we shall now
illustrate how sharp inversions in the subtropics can be dynamically extended
into the deep tropics.  For this idealized problem let us consider the initial
wind field $u=v=0$.  Since there is no initial rotational flow, $\Phi=\phi$
(or equivalently, $S=s$) and $\sigma^*=\sigma$ initially. Thus, we conclude
from (18) that $\sigma^*(\Phi,\Theta,\infty)=\sigma(\phi,\theta,0)$, which can
be substituted into the right hand side of (19).  For the initial mass field
we choose
  $$   \sigma(\phi,\theta,0) = \sigma_0 + \hat{\sigma} F(\theta)
       \cases{   1           &   $ \phi_w \le \phi \le  \pi/2 $   \cr
	         {1\over2}\left[1-\cos\left(\pi\phi/\phi_w\right)\right]
                             &   $-\phi_w \le \phi \le  \phi_w$   \cr
	         1           &   $-\pi/2  \le \phi \le -\phi_w$   \cr}
                                                                \eqno(25) $$
where
$$   F(\theta) = -{(\theta_2-\theta_1)\over 2(\theta_T-\theta_B)}
               + {1\over2}
       \cases{   0           &   $\theta_2 \le \theta \le  \theta_T$      \cr
	         1+\cos\left[\pi(2\theta-\theta_2-\theta_1)/
                                        (\theta_2-\theta_1)\right]
                             &   $\theta_1 \le \theta \le \theta_2$       \cr
	         0           &   $\theta_B \le \theta \le \theta_1$       \cr}
                                                                  \eqno(26) $$
Note that the vertical integral of (26) from $\theta_B$ to $\theta_T$
vanishes.
The constants appearing in (25) and (26) are specified as follows: 
$\theta_B=300$ K, $\theta_T=320$ K, $\theta_1=302$ K, $\theta_2=307$ K,
$\sigma_0=4.5$ kPa K$^{-1}$, $\hat{\sigma}=-4.4$ kPa K$^{-1}$, and
$\phi_w=\pi/9$. This choice of parameters results in a subtropical inversion
layer centered near 825 mb with a maximum stability of
$\partial T/\partial z=1.5$ Kkm$^{-1}$.

     The $p(\phi,\theta)$ field at the initial state and the $p(\phi,\theta)$
and $u(\phi,\theta)$ fields at the final adjusted state are shown in Fig.~8. 
The same information on the mass field is displayed in $\theta(\phi,p)$ plots
for the initial and final states in Fig.~9.  At the initial time there is an
inversion layer just below 800 mb in the subtropics but no inversion near the
equator.  This state is not in dynamic balance since there is no initial wind
but a nonzero initial variation of $\theta$ on a $p$-surface (or,
equivalently, an initial nonzero variation of $p$ on a $\theta$-surface).  To
reach dynamic equilibrium a westerly flow at the inversion level near the
equator is required.  The generation of such a westerly flow can be
accomplished by the poleward shift of air parcels near 800 mb in the
equatorial region.  By mass continuity and the adiabatic constraint, such
poleward shifts tend to fill in isentropic layers in subtropical latitudes
(decreasing the stability there) and evacuate isentropic layers near the
equator (increasing the stability there).  The end result is the dynamical
extension of the subtropical inversion into the tropical region.  The subtlety
of this process is reflected in the weak zonal winds produced ($\leq$ 1.3
ms$^{-1}$) and the small meridional parcel shifts involved (25 km poleward
shift to produce
1 ms$^{-1}$ westerly zonal flow at 15N).  Even though the zonal winds and
meridional particle shifts are small, the effect on the static stability field
is significant---a reflection of the general rule of dynamic adjustment of the
mass field to the wind field in the tropical region. The lapse rates for the
initial and final states at the equator and $30^\circ$ latitude are shown in
Fig.~10.  The static stability in the inversion at $30^\circ$ latitude
decreases while a meridionally extended stable layer appears at the equator.

Varying the strength of the subtropical stable layer from -4 K/km to 5 K/km
(results not shown) has only a minor effect on the zonal wind (+0.3 m/s at
10N) and stability (+0.5 K/km at the equator).

     It is possible to interpret this idealized experiment in a different
sense.  Imagine convective and radiative processes which operate to sharpen
the inversion at 800 mb in subtropical latitudes but not near the equator.  A
sequence of slow dynamical adjustments then results in a poleward drift of
equatorial air along isentropic surfaces extending into the developing
inversion.  This tends to weaken the subtropical inversion and strengthen the
tropical inversion, with the end result again being a dynamical extension of
inversion structure into the deep tropics.


\subsection{Experiment 2: adjustment of a sloping trade inversion layer}

     Let us now consider a second idealized problem which illustrates the
dynamical adjustments opposing the formation of sharp, sloping inversions in
the tropics.  For this second problem let us also consider the initial
condition $u=v=0$, so that again 
$\sigma^*(\Phi,\Theta,\infty)=\sigma(\phi,\theta,0)$.  For this case we choose
to specify $p(\phi,\theta,0)$, from which $\sigma(\phi,\theta,0)$ is easily
calculated by differentiation.  The initial pressure field is given by
  $$   p(\phi,\theta,0) = p_B - \sigma_0 (\theta-\theta_B) + \Delta p
          \cases{1    &   $\theta_2(\phi) \le \theta \le \theta_T$       \cr
	         G\left({\theta-\theta_1\over\theta_2-\theta_1}\right)
		      &   $\theta_1(\phi) \le \theta \le \theta_2(\phi)$ \cr
	         0    &   $\theta_B \le \theta \le \theta_1(\phi)$       \cr}
                                                                  \eqno(27) $$
where
$$   \theta_1(\phi) = 303K +  4K\,
       \cases{   0           &   $\phi_n \le \phi \le  \pi/2$       \cr
                 G\left({\phi_n-\phi\over\phi_n}\right)
                             &   $    0  \le \phi \le \phi_n$       \cr
	         G\left({\phi_s-\phi\over\phi_s}\right)
                             &   $\phi_s \le \phi \le    0  $       \cr
	         0           &   $-\pi/2 \le \phi \le \phi_s$       \cr}
                                                                  \eqno(28) $$
with $\theta_2(\phi)=\theta_1(\phi)+3K$ and
$ G(x)=\exp\left\{-(\gamma/x)\exp\left[1/(x-1)\right]\right\}$
an interpolating function.  Note that $G(0)=0$, $G(1)=1$, and with the choice
$\gamma={1\over2}\exp(2)\ln(2)$, $G({1\over2})={1\over2}$.  The constants
appearing in (27) and (28) are specified as follows: $p_B=100$ kPa,
$\theta_B=300$ K, $\theta_T=320$ K,
$\sigma_0=4.5$ kPa K$^{-1}$, $\Delta p=5.0$ kPa, $\phi_s=-\pi/8$, 
$\phi_n=\pi/8$.

     The fields of $\theta(\phi,p)$ at the initial and final states are shown
in Fig.~11. The corresponding zonal flows (not shown) are, as in the previous
experiment (cf. Fig.~8), easterly above the inversion and westerly below the
inversion. The final adjusted zonal winds are again weak, with the maximum
values just exceeding 1 ms$^{-1}$.  On the $\theta=304$ K surface, equatorial
air shifts poleward and fills mass into the flat subtropical inversion.  In
contrast, on the $\theta=308$ K surface, air shifts equatorward and fills mass
into the flat section of the equatorial inversion. On an intermediate surface,
$\theta=306$ K say, air at $20^\circ$N and $20^\circ$S shifts equatorward
while air at $5^\circ$N and $5^\circ$S shifts poleward, which fills mass into
the sharp, sloping inversion.  The end result is a significant weakening of
the flat section of the equatorial inversion near 650 mb, the formation of an
equatorial inversion near 800 mb, and the destruction of the sharpness of the
original sloping section of the inversion.  Note that it is not the overall
slope of the trade inversion which is decreased during adjustment but rather
the slope of each isentropic surface. The lapse rates for the initial and
final
states at the equator and $30^\circ$N are shown in Fig.~12. As in  experiment
1, this shows that the stability increase at the equator near 800mb in the
adjusted mass field is $\approx$20\% of the initial stability perturbation
imposed in the subtropics.  Finally, it is worth noting that the equatorial
stable layer is 25 mb higher than the subtropical stable layer, which yields a
slope in approximate agreement with observations.

     Another physical interpretation of these results is as follows.
Attempts by equatorial diabatic processes to reset the inversion to a higher
level are opposed, with the result that the higher inversion is weakened and
the lower inversion is reestablished.  Of course, these results do not imply
that it is impossible to form a sharp, sloping inversion in the deep tropics. 
Rather, they imply that it is difficult or ``inefficient" to form sharp
sloping inversions because of large scale dynamical constraints.

\vfill
\eject

\section{Concluding remarks}

     Through the use of a zonally symmetric model we have shown how the trade
wind inversion layer can be viewed as a structure which is dynamically
constrained to be more horizontally uniform than pure thermodynamical
arguments would suggest.  The misconception generated by purely thermodynamic
models of the horizontally homogeneous type is that the height of the trade
inversion is controlled by the {\it local values} of sea-surface temperature,
divergence and atmospheric temperature and moisture above the inversion.  The
dynamical view brings in the notion that the trade inversion layer is strongly
coupled horizontally.  Rapid variations in trade inversion height are
dynamically smoothed.  In this way, the trade inversion height is controlled
by horizontally averaged (over a Rossby length or more) values of the
sea-surface temperature, divergence and above-inversion atmospheric structure.

     It is sometimes argued that boundary layer processes can be parameterized
in general circulation models in a purely thermodynamic manner, i.e., by
appending equations which predict the boundary layer depth and thermodynamic
structure.  Such arguments should be viewed with caution because the model may
then have more vertical degrees of freedom in the mass field than the wind
field.  This means that part of the mass field is out of control of the
dynamical adjustment process.  Effects such as the ones we have discussed in
this paper may then be improperly simulated.

     From the observational results summarized in section 1, it is clear that
additional research should be conducted with the goal of producing monthly or
seasonal distributions of trade-wind inversion height over the entire tropical
and subtropical regions.  New tools such as 915 MHz wind profilers are capable
of continuously monitoring the trade inversion (Rogers et al.~1993) and could
be used in such future studies.

     Future theoretical and modeling research might involve the construction
of trade-wind boundary layer theories which can be studied in a pure
thermodynamic mode or a coupled dynamic/ther\-mo\-dy\-nam\-ic mode. 
Comparison of the results from these two modes of model implementation would
give better understanding of the basic concept presented here, i.e., the
concept that dynamical coupling leads to a trade inversion height which is
fairly uniform over the tropical and subtropical regions.

%$\theta$-surfaces are like porous membranes of variable stiffness.


\acknowledgments   This research was supported by the Office of Naval
Research, under grant N00014-91-J-1422.

%We wish to thank Richard Johnson for sharing with us his many physical %insights on tropical meteorology problems.

\vfill
\eject





\beginreferences


Albrecht, B. A., 1979:  A model for the thermodynamic structure of the
trade-wind booundary layer. II: Applications.  {\sl J.\ Atmos.\ Sci.},
{\bf 36}, 90--98.

%Albrecht, B. A., 1981:  Parameterization of trade-cumulus cloud amount. 
%{\sl J.\ Atmos.\ Sci.}, {\bf 38}, 73--89.

Albrecht, B. A., 1984:  A model study of downstream variations of the
thermodynamic structure of the trade winds.  {\sl Tellus},
{\bf 36A}, 187--202.

Albrecht, B. A., A. K. Betts, W. H. Schubert and S. K. Cox, 1979:  A model for
the thermodynamic structure of the trade-wind boundary layer. I: Theoretical
formulation and sensitivity tests.  {\sl J.\ Atmos.\ Sci.},
{\bf 36}, 73--89.

Augstein, E., H. Schmidt and F. Ostapoff, 1974: The vertical structure of the
atmospheric planetary boundary layer in undisturbed trade winds over the
Atlantic Ocean. {\sl Boundary-Layer Meteorology}, {\bf 6}, 129--150.

Betts, A. K., 1973: Nonprecipitating cumulus convection and its
parameterization. {\sl Quart.\ J.\ Roy.\ Meteor.\ Soc.}, {\bf 99}, 178--196.

Betts, A. K., and B. A. Albrecht, 1987:  Conserved variable analysis of the
convective boundary layer thermodynamic structure over the tropical oceans. 
{\sl J.\ Atmos.\ Sci.}, {\bf 44}, 83--99.

Bretherton, C. S., 1993:  Understanding Albrecht's model of trade cumulus
cloud fields.  {\sl J.\ Atmos.\ Sci.}, {\bf 50}, 2264--2283.

%Charney, J. G., and M. E. Stern, 1962:  On the stability of internal
%baroclinic jets in a rotating atmosphere.  {\sl J. Atmos.\ Sci.}, {\bf 19},
%159--172.

%Eliassen, A., 1983: The Charney--Stern theorem on barotropic-baroclinic
%instability.  {\sl Pure Appl.\ Geophys.}, {\bf 121}, 563--572.

Ficker, H. von, 1936: Die passatinversion. {\sl Ver\"of.\ des Meteor.\ Inst.\
der Univ.\ Berlin}, {\bf 1}, No.~4.

Firestone, J. K., and B. A. Albrecht, 1986:  The structure of the atmospheric
boundary layer in the central equatorial Pacific during January and February
of FGGE.  {\sl Mon.\ Wea.\ Rev.}, {\bf 114}, 2219--2231.

Gutnick, M., 1958: Climatology of the trade-wind inversion in the Caribbean.
{\sl Bull.\ Amer.\ Meteor.\ Soc.}, {\bf 39}, 410--420.

%Gutzler, D. S., 1993: Uncertainties in climatological tropical humidity %profiles: Some implications for estimating the greenhouse effect. {\sl J.\ %Climate}, {\bf 6}, 978--982.

%Hack, J. J., and W. H. Schubert, 1990:  Some dynamical properties of idealized
%thermally-forced meridional circulations in the tropics.  {\sl Meteorology and
%Atmospheric Physics}, {\bf 44}, 101--117.

%Hack, J. J.,  W. H. Schubert, D. E. Stevens and H.-C. Kuo, 1989:  Response of
%the Hadley circulation to convective forcing in the ITCZ.
%{\sl J.\ Atmos.\ Sci.}, {\bf 46}, 2957--2973.

%Haraguchi, P. Y., 1968: Inversions over the tropical eastern Pacific Ocean.
%{\sl Mon.\ Wea.\ Rev.}, {\bf 96}, 177--185.

Holland, J. Z., and E. M. Rasmusson, 1973:  Measurements of the atmospheric
mass, energy, and momentum budgets over a 500-kilometer square of tropical
ocean.  {\sl Mon.\ Wea.\ Rev.}, {\bf 101}, 44--55.

Hoskins, B. J., M. E. McIntyre and A. W. Robertson, 1985: On the use and
significance of isentropic potential vorticity maps.
{\sl Quart.\ J.\ R.\ Met.\ Soc.}, {\bf 111}, 877--946.

%Johnson, R. H., 1993:  Midlatitude convective systems.  NCAR Colloquium on %Clouds and Climate.

Johnson, R. H., J. F. Bresch, P. E. Ciesielski and W. A. Gallus, 1993: The
TOGA/COARE atmospheric sounding array: Its performance and preliminary
scientific results.  Preprints of the 20$^{\rm th}$ Conference on Hurricanes
and Tropical Meteorology, San Antonio, Texas, Amer.\ Meteor.\ Soc., 1--4.

Kloesel, K. A., and B. A. Albrecht, 1989:  Low-level inversions over the
tropical Pacific---ther\-mo\-dy\-nam\-ic structure of the boundary layer and
the above-inversion moisture structure. {\sl Mon.\ Wea.\ Rev.}, {\bf 117},
87--101.

Lilly, D. K., 1968:  Models of cloud-topped mixed layers under a strong
inversion. {\sl Quart.\ J.\ Roy.\ Meteor.\ Soc.}, {\bf 94}, 292--309.

%Magnusdottir, G., and W. H. Schubert, 1991:  Semigeostrophic theory on the
%hemisphere. {\sl J. Atmos.\ Sci.}, {\bf 48}, 1449--1456.

Nitta, T., and S. Esbensen, 1974: Heat and moisture budget analyses using
BOMEX data.  {\sl Mon.\ Wea.\ Rev.}, {\bf 102}, 17--28.

Ramage, C. S., S. J. S. Khalsa and B. N. Meisner, 1981:  The central Pacific
near-equatorial convergence zone. {\sl J.\ Geophys.\ Res.}, {\bf 86},
6580--6598.

Randall, D. A., 1980: Entrainment into a stratocumulus layer with distributed
radiative cooling. {\sl J. Atmos.\ Sci.}, {\bf 37}, 148--159.

Rogers, R. R., W. L. Ecklund, D. A. Carter, K. S. Gage and S. A. Ethier, 1993:
 Research applications of a boundary-layer wind profiler. {\sl Bull.\ Amer.\ 
Meteor.\ Soc.}, {\bf 74}, 567--580.

Sarachik, E. S., 1985: A simple theory for the vertical structure of the
tropical atmosphere. {\sl PAGEOPH}, {\bf 123}, 261--271.

Schubert, W. H., P. E. Ciesielski, D. E. Stevens and H.-C. Kuo, 1991:
Potential vorticity modeling of the ITCZ and the Hadley circulation.
{\sl J. Atmos.\ Sci.}, {\bf 48}, 1493--1509.

Schubert, W. H., P. E. Ciesielski, T. B. McKee, J. D. Kleist, S. K. Cox, C. M.
Johnson-Pasqua, and W. L. Smith Jr., 1987:  Analysis of Boundary Layer
Sounding Data from the FIRE Marine Stratocumulus Project. Colorado State
University, Department of Atmospheric Science, Report No. 419, 97 pp.

Schubert, W. H., S. K. Cox, T. B. McKee, D. A. Randall, P. E. Ciesielski, J.
D. Kleist, and E. L. Stevens, 1992:  Analysis of Sounding Data from Porto
Santo Island During ASTEX. Colorado State University, Department of
Atmospheric Science, Report No. 512, 96 pp.

Schubert, W. H., J. S. Wakefield, E. J. Steiner and S. K. Cox, 1979a: Marine
stratocumulus convection. Part I: Governing equations and horizontally
homogeneous solutions. {\sl J.\ Atmos.\ Sci.}, {\bf 36}, 1286--1307.

Schubert, W. H., J. S. Wakefield, E. J. Steiner and S. K. Cox, 1979b: Marine
stratocumulus convection. Part II:  Horizontally inhomogeneous solutions.
{\sl J.\ Atmos.\ Sci.}, {\bf 36}, 1308--1324.

%Shutts, G. J., 1980:  Angular momentum coordinates and their use in zonal,
%geostrophic motion on a hemisphere. {\sl J. Atmos.\ Sci.}, {\bf 37},
%1126--1132.

Wakefield, J. S., and W. H. Schubert, 1981:  Mixed-layer model simulation of
Eastern North Pacific stratocumulus.  {\sl Mon.\ Wea.\ Rev.}, {\bf 109},
1959--1968.


\endreferences


\begincaptions


Fig. 1. Cumulative frequency (percent) of lapse rates greater than $-4$ K
km$^{-1}$ (solid curve), $0$ K km$^{-1}$ (dashed curve) and $4$ K km$^{-1}$
(dotted curve).

Figure 2.  Sounding statistics for 70 high resolution radiosonde soundings at 
San Nicolas Island ($33.3^\circ$N, $119.6^\circ$W) from 30 June to 20 July
1987 in a format identical to Fig.~1.

Figure 3. Height (m) of the base of the trade-wind inversion over the
Atlantic,  as determined from data obtained during the 1925--27 Meteor I
expedition. (From von Ficker 1936).

Figure 4. Height (m) of the base of the trade-wind inversion over the
Caribbean for April.  The map is based on the tabular data of Gutnick (1958).

Figure 5.  Mean north-south cross section of saturation equivalent potential
temperature (degrees Celsius) near $165^\circ$W based on dropwindsonde
soundings taken between 15 January and 20 February 1979.  The dashed curve
indicates the inversion height based on the minimum saturation equivalent
potential temperature.  (From Firestone and Albrecht 1986).

Figure 6.  Sounding statistics for Kwajelein ($8.7^\circ$N, $167.7^\circ$E),
in a format identical to Fig.~1.

Figure 7. Schematic north-south cross section depicting a steeply sloped trade
inversion (dashed line).

Figure 8. Results for experiment 1: (a) isoline of $p(\phi,\theta)$ for the
initial unbalanced state: (b) isolines of $p(\phi,\theta)$ and
$u(\phi,\theta)$
for the final balanced state. Solid wind contours indicate westerly flow,
dashed contours easterly flow, with a contour interval of 0.25 ms$^{-1}$.

Figure 9. Results for experiment 1: (a) isolines of $\theta(\phi,p)$ for the
initial unbalanced state;
(b) isolines of $\theta(\phi,p)$ for the final balanced state.

Figure 10.  Lapse rate $\partial T/\partial z$ as a function of pressure for
the initial state (solid curves) and the final adjusted state (dashed curves)
at the equator (panel a) and at $30^\circ$ latitude (panel b) for experiment
1.  Initially there is a stable layer at $30^\circ$ latitude but not at the
equator.  In the final state the stable layer has been dynamically extended to
the equator.

Figure 11. Same as in Fig.~9, but for experiment 2.

Figure 12. Same as in Fig.~10, but for experiment 2.

\endcaptions




\bye

********  Extra Pieces  ****************

     Kloesel and Albrecht (1989) studied approximately 1200 dropwindsonde
soundings taken between 5S and 15N during 15 January--20 February 1979 (FGGE
SOP I) and 10 May--8 June 1979 (FGGE SOP II).  They concluded that the mean
trade inversion over the eastern Pacific has an inversion top at 800 mb and a
base at 850 mb, and that there is little latitudinal or longitudinal variation
in the height of the inversion.  These conclusions are consistent with the
earlier dropwindsonde study of Firestone and Albrecht (1986), who used a
smaller dataset, and with the aircraft study of Ramage et al.\ (1981).



    To express the invertibility principle in terms of $p$ we differentiate
(8) with respect to $\theta$ and use the thermal wind equation to obtain
  $$  {\partial \over a\cos\phi\partial\phi}
       \left({\Gamma\cos\phi\over\hat{f}}{\partial p\over
a\partial\phi}\right)
       + P {\partial^2 p \over\partial\theta^2}
       = \sigma {\partial P \over \partial\theta},       \eqno(12{\rm a}) $$
with boundary conditions
  $$  p=p_T \;\;\; {\rm at} \;\;\;\theta = \theta_T,     \eqno(12{\rm b}) $$
  $$  \theta_B{\sin\phi\over\cos^3\phi}{\partial\over a\partial\phi}
       \left({\Gamma\cos^3\phi\over\sin\phi}{\partial p
                                             \over a\partial\phi}\right)
      + \hat{f}P{\partial p \over \partial\theta}=
       -4\Omega^2\sin^2\phi
       \; \; \; {\rm at} \; \; \;\theta = \theta_B,      \eqno(12{\rm c}) $$
  $$    {\partial p \over \partial\phi} = 0
       \;\;\;{\rm at}\;\;\;\phi =\pm{\pi\over2}.         \eqno(12{\rm d}) $$


      To express the invertibility principle in terms of $M$ we multiply (9)
by $(\cos^3\phi)/(\sin\phi)$ and differentiate with respect to $\phi$ to
obtain
  $$  {\sin\phi \over \cos^3\phi}{\partial \over a\partial\phi}
       \left({\cos^3\phi\over\sin\phi}{\partial M \over a\partial\phi}\right)
       + {\hat{f}P \over\Gamma}{\partial^2 M \over\partial\theta^2}
       = -4\Omega^2\sin^2\phi,                           \eqno(13{\rm a}) $$
with boundary conditions
  $$ {\partial M \over \partial\theta}=\Pi_T
       \;\;\; {\rm at} \;\;\;\theta = \theta_T,          \eqno(13{\rm b}) $$
  $$ {\partial M \over \partial\theta}={M\over\theta_B}
       \; \; \; {\rm at} \; \; \;\theta = \theta_B,      \eqno(13{\rm c}) $$
  $$    {\partial M \over \partial\phi} = 0
       \;\;\;{\rm at}\;\;\;\phi =\pm{\pi\over2}.         \eqno(13{\rm d}) $$



     To express the invertibility principle in terms of $p$ we differentiate
(8) with respect to $\Theta$ and use the thermal wind equation to obtain
  $$  {\partial \over a\cos\Phi\partial\Phi}
       \left({\Gamma\cos\Phi\over\hat{f}}{\partial p\over
a\partial\Phi}\right)
       + P {\partial^2 p \over\partial\Theta^2}
       = {\partial \sigma^*2\Omega\sin\phi \over \partial\Theta},
                                                         \eqno(12{\rm a}) $$
with boundary conditions
  $$  p=p_T \;\;\; {\rm at} \;\;\;\Theta = \Theta_T,     \eqno(12{\rm b}) $$
  $$  \theta_B{\sin\phi\over\cos^3\phi}{\partial\over a\partial\phi}
       \left({\Gamma\cos^3\phi\over\sin\phi}{\partial p
                                             \over a\partial\phi}\right)
      + \hat{f}P{\partial p \over \partial\theta}=
       -4\Omega^2\sin^2\phi
       \; \; \; {\rm at} \; \; \;\Theta = \Theta_B,      \eqno(12{\rm c}) $$
  $$    {\partial p \over \partial\Phi} = 0
       \;\;\;{\rm at}\;\;\;\Phi =\pm{\pi\over2}.         \eqno(12{\rm d}) $$

\section{Solutions of the potential pseudodensity equation for melting layer
forcing}

     Let us consider the case where $\dot{\theta}$ is independent of time and
has a spatial dependence given by
  $$   \dot{\Theta}(S,\Theta) = -Q(S)
         \cases{1+\cos(\pi Z)   & $-1 \le Z \le 1$  \cr
                0               &  otherwise        \cr}           \eqno(1) $$
where $Z=(2\Theta-\Theta_2-\Theta_1)/(\Theta_2-\Theta_1)$, $\Theta_1$ and 
$\Theta_2$ are constants, and $Q(S)$ is the latitudinal distribution of the
heating.  Note that, since
$(\Theta_2-\Theta_1)^{-1}\int_{\Theta_1}^{\Theta_2}\dot{\Theta}
d\Theta=-Q(S)$, we can regard $-Q(S)$ as the vertically averaged value of
$\dot{\Theta}$ in the melting layer.  Multiplying () by $\dot{\theta}$ we
obtain
  $$  {\partial(\dot{\theta}\sigma^*) \over \partial\tau}
    - \left[1 + \cos(\pi Z)\right]
     {\partial(\dot{\theta}\sigma^*) \over \partial Z} = 0,
                                                                   \eqno(2) $$
where $\tau=2Q(S)T/(\Theta_2-\Theta_1)$.

According to (2) the quantity $\dot{\theta}\sigma^*$ is constant along each
characteristic curve determined from $-dZ/[1+\cos(\pi Z)]=d\tau$.
By integration of this equation we can show that the characteristic through
the point $(Z,\tau)$ intersects the $\tau=0$ axis at a level $Z_0(Z,\tau)$
determined by $\pi Z_0(Z,\tau)=2\tan^{-1}[\pi\tau+\tan({1\over2}\pi Z)]$. 
Since $\dot{\theta}\sigma^*$ is constant along each characteristic, its value
at $(Z,\tau)$ must equal its value at $(Z_0,0)$, which results in
  $$  \sigma^*(Z,\tau) = \sigma_0
      \left({1 + \cos\{2\tan^{-1}[\pi\tau+\tan(\textstyle{{1\over2}}\pi Z)]\} 
       \over 1 + \cos(\pi Z)}\right).
                                                                   \eqno(3) $$
For the latitudinal distribution of the cooling due to melting we choose the
form
  $$   Q(S) = {4\alpha Q_0 \over \pi^{1/2}}
              \left({\exp[-\alpha^2(S-S_c)^2] \over
              {\rm erf}[\alpha(1+S_c)] + {\rm erf}[\alpha(1-S_c)]}\right),
                                                                   \eqno(4) $$
where $S_c$ and $\alpha$ are parameters which control the latitude and width
of the convective zone.  Because of the way the product $Q(S)T$ appears in the
definition of $\tau$, it is not necessary to choose $Q_0$.  The solution can
simply be obtained for different values of $Q_0T$.  However, for purposes of
physical interpretation let us choose $Q_0=0.30$ K day$^{-1}$, along with
$\Theta_B=300$ K and $\Theta_T=360$ K.

\centerline{Uniqueness of the physically acceptable solution to the
invertibility principle}

     Assume that the solutions ${\cal M}_1$ and ${\cal M}_2$ exist and that
both solutions satisfy the invertibility principle () and its boundary
conditions ().
  $$   A {\partial^2{\cal M}_d \over \partial S^2}
     +2B {\partial^2{\cal M}_d \over \partial S^2}
     + C {\partial^2{\cal M}_d \over \partial\Theta^2} = 0,
                                                           \eqno({\rm A}.1) $$

      To express the invertibility principle in terms of ${\cal M}$ we
multiply (9) by $(\cos^3\phi)/(\sin\phi)$ and differentiate with respect to
$\phi$ to obtain
  $$ \eqalignno{ \left[4\Omega^2\sin^2\Phi
                 &-{\sin\Phi \over \cos\Phi}{\partial \over a\partial\Phi}
       \left({\cos^2\phi\over\cos\Phi\sin\Phi}
                       {\partial{\cal M}' \over a\partial\Phi}\right)\right]
       \left[\bar{\Gamma}\bar{\sigma}
                      -{\partial^2{\cal M}'\over\partial\Theta^2}\right] \cr	 
     &- \left({\partial^2{\cal M}' \over a\partial\Phi\partial\Theta}\right)
         {\partial\over\partial\Theta}\left({\cos^2\phi\over\cos^2\Phi}
	         {\partial{\cal M}' \over a\partial\Phi}\right)
       = \Gamma\sigma^* 4\Omega^2\sin\Phi\sin\phi, &(13{\rm a}) \cr} $$
with boundary conditions
  $$ {\partial{\cal M}'\over\partial\Theta}=0
       \;\;\; {\rm at} \;\;\;\Theta = \Theta_T,          \eqno(13{\rm b}) $$
  $$ {\partial {\cal M}' \over \partial\Theta}={M'\over\Theta_B}
       \; \; \; {\rm at} \; \; \;\Theta = \Theta_B,      \eqno(13{\rm c}) $$
  $$    {\partial {\cal M} \over \partial\Phi} = 0
       \;\;\;{\rm at}\;\;\;\Phi =\pm{\pi\over2}.         \eqno(13{\rm d}) $$

  $$ {\partial s \over \partial S}
     {\partial^2 {\cal M} \over \partial\Theta^2}
   - {\partial s \over \partial\Theta}
     {\partial^2 {\cal M} \over \partial S \partial\Theta}
   + \Gamma \sigma^* = 0,                                \eqno(19{\rm a}) $$
  $$ 2\Omega^2 a^2 S
      \left({s^2-S^2 \over 1-s^2}\right)
     + {\partial {\cal M} \over \partial S} = 0.         \eqno(19{\rm b}) $$
Equations (14a--b) constitute the desired relation between ${\cal M}$, $s$
and $\sigma^*$.   For boundary conditions we choose
  $$ {\partial {\cal M} \over \partial\Theta}=\Pi_T \; \; \; {\rm at} \; \; \;
            \Theta = \Theta_T,    \qquad
      \Theta {\partial {\cal M} \over \partial \Theta} - {\cal M}
      + {\Omega^2 a^2 (s^2-S^2)^2 \over 2(1-s^2)} = 0
       \; \; \; {\rm at} \; \; \;\Theta = \Theta_B,      \eqno(14{\rm c}) $$
  $$    s = 1 \;\;\;{\rm at}\;\;\;S = 1,   \qquad
        s =-1 \;\;\;{\rm at}\;\;\;S =-1.                 \eqno(14{\rm d}) $$
Equation (14c) results from assuming that the upper isentropic surface
$\Theta = \Theta_T$ is also an isobaric surface with Exner function $\Pi_T$.
The lower boundary condition results from assuming the geopotential vanishes
on the lower isentropic surface $\Theta = \Theta_B$, so that $M = \Theta \Pi$
there.  Then, expressing $M$ in terms of ${\cal M}$ and $s$, we can write
the lower boundary condition as (14c).  For the boundary conditions at the
poles, symmetry requires the conditions (14d).

     To isolate the dynamically significant portion of ${\cal M}$ we proceed
as follows.  We define $\bar{\Pi}$, $\bar{\Gamma}$, $\bar{\sigma}$ and
$\bar{\cal M}$ by
  $$   \bar{\Pi} = c_p \left({\bar{p} \over p_0}\right)^\kappa, \qquad
       \bar{\Gamma} = {d\bar{\Pi} \over d\bar{p}}
                    = {\kappa\bar{\Pi}\over\bar{p}},            \qquad
       \bar{\sigma} = - {d\bar{p} \over d\Theta},               \qquad
       \bar{\cal M} = \Theta_B\bar{\Pi}_B
                    + \int_{\Theta_B}^\Theta \bar{\Pi}d\Theta',           $$
with $\bar{p}$ given by
  $$   \bar{p}(\Theta) = p_T  + {\textstyle{1\over2}}
       \int_\Theta^{\Theta_T}\int_{-\pi/2}^{\pi/2}
         \sigma^* \cos\Phi\,d\Phi\,d\Theta'.
	                                                        \eqno(14) $$
  $$ {\partial s \over \partial S}
     {\partial^2 {\cal M} \over \partial\Theta^2}
   - {\partial s \over \partial\Theta}
     {\partial^2 {\cal M} \over \partial S \partial\Theta}
   + \Gamma \sigma^* = 0,                                \eqno(14{\rm a}) $$
  $$ 2\Omega^2 a^2 S
      \left({s^2-S^2 \over 1-s^2}\right)
     + {\partial {\cal M} \over \partial S} = 0,         \eqno(14{\rm b}) $$
where $\Gamma=\kappa\Pi/p$.
Equations (3.14a--b) constitute the desired relation between ${\cal M}$, $s$
and $\sigma^*$.   For boundary conditions we choose
  $$ {\partial {\cal M}' \over \partial\Theta}=0 \; \; \; {\rm at} \; \; \;
            \Theta = \Theta_T,                           \eqno(14{\rm c}) $$
  $$  \Theta {\partial {\cal M} \over \partial \Theta} - {\cal M}
      + {\Omega^2 a^2 (s^2-S^2)^2 \over 2(1-s^2)} = 0
       \; \; \; {\rm at} \; \; \;\Theta = \Theta_B,      \eqno(14{\rm d}) $$
  $$    s = 1 \;\;\;{\rm at}\;\;\;S = 1,                 \eqno(14{\rm e}) $$
  $$    s =-1 \;\;\;{\rm at}\;\;\;S =-1.                 \eqno(14{\rm f}) $$






 For the initial mass field we choose
  $$   \sigma(\phi,\theta,0) = \sigma_0 + \hat{\sigma}
               \left[1-e^{-(\phi/\phi_w)^2}\right] F(\theta),     \eqno(25) $$
where
$$   F(\theta) = -{(\theta_2-\theta_1)\over 2(\theta_T-\theta_B)}
               + {1\over2}
       \cases{   0           &   $\theta_2 \le \theta \le  \theta_T$      \cr
	         1+\cos\left[\pi(2\theta-\theta_2-\theta_1)/
                                        (\theta_2-\theta_1)\right]
                             &   $\theta_1 \le \theta \le \theta_2$       \cr
	         0           &   $\theta_B \le \theta \le \theta_1$       \cr}
                                                                  \eqno(26) $$
Note that the vertical integral of (26) from $\theta_B$ to $\theta_T$
vanishes.
The constants appearing in (25) and (26) are specified as follows: 
$\theta_B=300$ K, $\theta_T=320$ K, $\theta_1=306$ K, $\theta_2=310$ K,
$\sigma_0=2.8$ kPa K$^{-1}$, $\hat{\sigma}=-2.0$ kPa K$^{-1}$, and $\phi_w=20$
degrees.


    In one-dimensional thermodynamic models of stratocumulus (e.g., Wakefield
and Schubert 1981, Albrecht 1984) and shallow cumulus (e.g., Albrecht et al.\
1979, Albrecht 1979, Bretherton 1993) the equilibrium trade wind boundary
layer depth is determined by a balance between the radiative and moist
convective processes tending to deepen the layer and the large-scale
subsidence tending to shallow the layer.  These pure thermodynamic models
predict a deepening boundary layer along trajectories moving toward the ITCZ
over increasing sea surface temperature and decreasing large-scale subsidence.

     In possession of concepts like these, it is natural to construct
schematic cross sections such as that shown in Fig.~7, which depicts the trade
inversion as rising from very near the sea surface in the subtropics to
midtroposphere  near the edge of the ITCZ.  Evidence that such depictions are
inconsistent with observations has been provided by

      Three theoretical views of the trade inversion are as follows.  The
first considers the trade inversion as a pure thermodynamical phenomenon which
can be modeled using steady state, horizontally homogeneous assumptions.  This
view leads to the notion that the boundary layer depth and thermodynamic
structure are controlled by local values of sea surface temperature and
large-scale divergence.  The second view also involves pure thermodynamical,
steady state arguments, but recognizes the importance of horizontal
inhomogeneities.  In this view, horizontal advection is important and
entrainment can occur not only by large-scale subsidence but also by
horizontal flow across a sloping inversion.  This view leads to the notion
that the boundary layer depth and thermodynmic structure are controlled by a
weighted average of the sea surface temperature and large-scale divergence
upstream along the trajectory. The third view involves combined
dynamical/thermodynamical arguments.



